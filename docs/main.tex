\documentclass{article}
\usepackage[utf8]{inputenc}
\usepackage[spanish]{babel}
\usepackage{graphicx}
\usepackage{dirtytalk}
\usepackage{caratula}
\usepackage{enumerate}
\usepackage{amssymb}
\usepackage{amsmath}
\usepackage{geometry}
\usepackage{fixltx2e}
\usepackage{wrapfig}
\usepackage{cite}
\usepackage{float}
\usepackage[space]{grffile}
\geometry{
 a4paper,
 total={210mm,297mm},
 left=30mm,
 right=30mm,
 top=30mm,
 bottom=30mm,
 }
 
\begin{document}
% Estos comandos deben ir antes del \maketitle
\materia{Algorítmos y Estructuras de Datos III} % obligatorio

\titulo{Trabajo Práctico 1}
\subtitulo{"Knock, knock".- \\
"Who's there?".- \\
$\cdot \cdot \cdot \cdot \cdot \cdot \cdot \cdot \cdot \cdot \cdot \cdot \cdot \cdot \cdot \cdot \cdot \cdot \cdot \cdot \cdot \cdot \cdot \cdot \cdot \cdot \cdot$ \\
$\cdot \cdot \cdot \cdot \cdot \cdot \cdot \cdot \cdot \cdot \cdot \cdot \cdot \cdot \cdot \cdot \cdot \cdot \cdot \cdot \cdot \cdot \cdot \cdot \cdot \cdot \cdot$ \\
"Java".-}
\grupo{
En este informe se presenta la implementación, desarrollo y análisis de resultados usados en la resolución de un problema práctico, con las herramientas estudiadas y provistas por la materia. El mismo consiste en agrandar una imagen por un cierto factor $k$.
Palabras Clave: Vecinos Más Cercanos, Interpolación Bilineal, Interpolación por Splines, Zoom\\}

\integrante{Bayardo Julián}{850/13}{julian@bayardo.com.ar} % obligatorio
\integrante{Cuneo Christian}{755/13}{chriscuneo93@gmail.com} % obligatorio 
\integrante{Frassia Fernando}{340/13}{ferfrassia@gmail.com} % obligatorio 
\integrante{Gambaccini Ezequiel}{715/13}{ezequiel.gambaccini@gmail.com} % obligatorio 
 
\maketitle

\pagebreak

\tableofcontents

\pagebreak

\section{Introducción}

\section{Problema 1}

\subsection{Correctitud}

\subsection{Complejidad}

\subsection{Análisis}

\section{Problema 2}

\subsection{Correctitud}

\subsection{Complejidad}

\subsection{Análisis}

\section{Problema 3}

\subsection{Correctitud}

\subsection{Complejidad}

\subsection{Análisis}

\section{Conclusión}

\section{Código fuente}

\end{document}